\label{sec:corpus}
Annotating a corpus with coreference information is an expensive task, as every sentence should include at least one entity and event.  Since many of these mentions will refer to other mentions in the same sentence, document, or other documents, the task quickly becomes complex and time-consuming.  Further, many mentions can be tricky to perfectly annotate, and consequently, annotators will often differ in their markings.  Therefore, multiple annotators label every sentence, allowing a majority vote to resolve discrepancies.  Due to these difficulties, there are not many event corpora, yet the ECB+ is sufficient for research, and we now describe its evolution.

\section{Event Coreference Corpora}
\subsection{ECB: EventCorefBank}
Created by Bejan and Harabagiu in 2010 \cite{Bejan:2010:UEC:1858681.1858824}, the ECB corpus provides within-document and cross-document event coreference annotations for 480 documents, spanning 43 disjoint \textit{topics}.  The documents were selected from GoogleNews archive\footnote{http://news.google.com} and each topic is a collection of documents (roughly 7-15 relatively short documents) which all concern the same seminal event, such as a particular arrest, transaction, attack, sporting event, election, etc.  Each event mention, and its relation with another event, was annotated, where an even relation is one of six types: subevent, reason, purpose, enablement, precendence, and related.  The weakness of this corpus is that (1) only a subset of the sentences were annotated; (2) only events were annotated -- no entities.

\subsection{EECB: Extended ECB}
Lee, et. al. \cite{Lee:2012:JEE:2390948.2391006} extended the ECB corpus by addressing both of the aforementioned weaknesses; four annotators fully annotated all sentences, with entity coreference relations included.  Also, they removed the originally-annotated relations, only keeping the coreference ones (e.g., subevent, purpose, related).  Note, \textit{light verbs} were not annotated (e.g., \textit{make} an offer or \textit{give} a talk).

\subsection{ECB+: EventCorefBank+}
All of our event-based experiments were conducted on this corpus, as it is the largest and commonly used corpus.  It includes the 480 documents from the original ECB corpus, along with 502 additional documents which stem from 43 additional topics which are highly similar to the original 43 topics from the original ECB corpus.  For example, Topic 1 in the ECB corpus contains documents which concern Tara Reid checking into a rehab center in Malibu, California.  However, Topic 1 in the ECB+ corpus also includes documents which concern Lindsay Lohan checking into a rehab center in Rancho Mirage, California.  The purpose for this similarity is to help create a more realistic and potentially confusable scenario for the cross-document task.  However, it has been shown \cite{journals/tacl/YangCF15} that one can simply perform document classification as a pre-processing step, which will allow a cross-document model to appropriately, perfectly confine itself to a relevant document set from which to consider linking mentions (e.g., ECB's Topic 1 documents can be easily separable from ECB+'s added Topic 1 documents).

We maintain the same train/dev/test splits as previous researchers, as further detailed in Chapter \ref{sec:coreference}.  A sample of the corpus in shown in Figure \ref{fig:corpus}, and statistics are listed in Table \ref{tab:ECB1}, where it is clear that the majority of gold mentions are one token in length (e.g, \textit{announced}).  \textbf{NOTE:} the corpus contains 15,812 sentences.  The corpus creators only place stock in their annotations for 1,840 specific sentences.  However, they also annotate some additional sentences which they place less stock in (e.g., not all annotators labelled these additional sentences).  All research papers which use this corpus simply evaluate again all labelled mentions, even if they do not belong to the well-supported 1,840 sentences.  For fair comparison, we follow suit.

\begin{table}
\centering
\begin{tabular}{c|c|c|c|c|}
\cline{2-5}
& Train & Dev & Test & Total \\ \cline{1-5} \hline
\multicolumn{1}{ |c| }{\# Documents} & 462 & 73 & 447 & 982   \\ %\cline{1-5}
\multicolumn{1}{ |c| }{\# Sentences} & 7,294 & 649 & 7,867 & 15,810    \\ 
\multicolumn{1}{ |c| }{\# Mentions-1} & 1,938 & 386 & 2,837 & 5,161    \\ %\cline{1-5}
\multicolumn{1}{ |c| }{\# Mentions-2} & 142 & 52 & 240 & 434    \\ %\cline{1-5}
\multicolumn{1}{ |c| }{\# Mentions-3} & 18 & -- & 25 & 43    \\% \cline{1-5}
\multicolumn{1}{ |c| }{\# Mentions-4} & 6 & -- & 7 & 13   \\ \cline{1-5}
\end{tabular}
\caption{Statistics of the ECB+ Corpus, where Mentions-N represents event mentions which are N-tokens in length.}
\label{tab:ECB1}
\end{table}


%\section{Entity Coreference Corpora}
%Forthcoming